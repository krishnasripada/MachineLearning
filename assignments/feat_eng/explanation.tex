\documentclass[10pt]{article}
\usepackage{amsmath}
\usepackage{mathtools}
\DeclarePairedDelimiter{\abs}{\lvert}{\rvert}
\usepackage[hidelinks]{hyperref}
\usepackage{amssymb}
\usepackage{tikz}
\usepackage{caption}
\usepackage{graphicx}
\usepackage[T1]{fontenc}
\graphicspath{{.}}
\usepackage{listings}
\usepackage{verbatim}
\lstset{
language=[LaTeX]TeX,
backgroundcolor=\color{gray!25},
basicstyle=\ttfamily,
columns=flexible,
breaklines=true
}
\captionsetup{labelsep=space,justification=justified,singlelinecheck=off}
\reversemarginpar
\usepackage[paper=a4paper,
            %includefoot, % Uncomment to put page number above margin
            marginparwidth=20mm,      % Length of section titles
            marginparsep=0.8mm,       % Space between titles and text
            margin=12mm,              % 25mm margins
            includemp]{geometry}

\begin{document}
\section*{}
\begin{flushleft}
Name: Krishna Chaitanya Sripada\\
CU Identikey: krsr8608\\
\end{flushleft}
\section*{Explanation of Features: }
\begin{flushleft}
The process of creating additional features is:\\
\vspace{0.5em}
1. I've used the tokenizer parameter that can be passed to the CountVectorizer. This tokenizer parameter is a dictionary which is returned by the features I've used. \\
\vspace{0.5em}
2. English stopwords are removed by passing the ``stop\_words'' parameter to the CountVectorizer function.\\
\vspace{0.5em} 
3. Feature-1: Bag of words is used where the count of each word present in the data is maintained. \\
\vspace{0.5em}
4. Feature-2: A custom function is written to generate the bigrams where the input for this feature is a tokenized list of the ``Text'' column of the training data.\\
\vspace{0.5em}
5. The tokenizer used here is the ``WordPunctTokenizer'' which takes the data and uses regular expression to tokenize the data.\\
\end{flushleft}
\end{document}